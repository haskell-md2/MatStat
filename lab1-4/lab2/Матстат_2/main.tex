\documentclass[12pt]{article}
\usepackage[utf8x]{inputenc}
\usepackage[english,russian]{babel}
\usepackage{cmap}
\usepackage{amsmath}
\usepackage{bm}
\usepackage{hyperref}
\usepackage{enumitem}
\usepackage{geometry} % Меняем поля страницы
\geometry{left=1.5cm}% левое поле
\geometry{right=2cm}% правое поле
\geometry{top=1cm}% верхнее поле
\geometry{bottom=2cm}% нижнее поле
\emergencystretch 3em

\usepackage{graphicx}%Вставка картинок правильная
\usepackage{float}%"Плавающие" картинки
\usepackage{wrapfig}%Обтекание фигур (таблиц, картинок и прочего)
\usepackage{amsfonts} 
\usepackage{mathtools}

\ExplSyntaxOn
\NewDocumentCommand{\pyf}{m}
 {
  \texttt
   {
    \tl_set:Nn \l_tmpa_tl { #1 }
    \tl_replace_all:Nen \l_tmpa_tl { \char_generate:nn { `_ } { 8 } } { \_ }
    \tl_use:N \l_tmpa_tl
   }
 }
\cs_generate_variant:Nn \tl_replace_all:Nnn { Ne }
\ExplSyntaxOff

\newtheorem{theorem}{Теорема}

\newcommand{\fo}[1]{$\boldsymbol{#1}$}
\newcommand{\Fo}[1]{$$\boldsymbol{#1}$$}

\newcommand{\ro}[1]{%
  \xrightarrow{\mathmakebox[\rowidth]{#1}}%
}



\begin{document}

    \begin{titlepage}
    \LARGE
        \begin{center}
            Санкт-Петербургский Политехнический \\ университет имени Петра Великого

            \vspace{8em}
            
            Отчёт по лабораторной работе №2 \\
            \vspace{1em}
            \textbf{Характеристики положения и рассеяния}
            \end{center}
            \vspace{9em}
            \vspace{\fill}
            Студент: Расторгуев Михаил Павлович \\
            Группа: 5030102/30201

            \vspace{5em}
            
            \begin{center}
            Санкт-Петербург \\2026
            \end{center}
       

    \end{titlepage}

    \newpage

    \large
    
    \section{Постановка задачи}
    
    В рамках данной лабораторной работы было необходимо:
    \begin{itemize}
        \item Исследовать сходимость выборочных характеристик к теоретическим при росте n.
        \item Исследовать оценки характеристик положения на устойчивость к выбросам.
    \end{itemize}
    
    \section{Теоретическая часть}
    В данной работе требовалось изучить следующие распределения случайных величин:
        \begin{itemize}
        \item Нормальное распределение \( N(x; 0, 1) \)
        \[
        f(x) = \frac{1}{\sqrt{2\pi}} e^{-\frac{x^2}{2}}, \quad x \in \mathbb{R}
        \]
    
        \item Распределение Коши \( C(x; 0, 1) \)
        \[
        f(x) = \frac{1}{\pi(1 + x^2)}, \quad x \in \mathbb{R}
        \]
    
        \item Распределение Лапласа \( L\left(x; 0, \frac{1}{\sqrt{2}}\right) \)
        \[
        f(x) = \frac{1}{\sqrt{2}} e^{-\sqrt{2}|x|}, \quad x \in \mathbb{R}
        \]
    
        \item Распределение Пуассона \( P(k; 5) \)
        \[
        P(X = k) = \frac{5^k e^{-5}}{k!}, \quad k = 0, 1, 2, \dots
        \]
    
        \item Равномерное распределение \( U(x; -\sqrt{3}, \sqrt{3}) \)
        \[
        f(x) = \frac{1}{2\sqrt{3}}, \quad x \in [-\sqrt{3}, \sqrt{3}]
        \]
    \end{itemize}
    
    \section{Реализация}
        Практическая часть работы была реализована на языке программирования \texttt{ Python} с использованием библиотек: \texttt{scipy,seaborn,numpy,pandas}.\\
        Код писался в интерактивном блокноте \texttt{Jupyter Notebook}.\\\\
        Были реализованы следующие функции:\\\\
        \pyf{get_five_values(n,dist,**param)} - функция, генерирующая набор из случайных данных и затем считает требуемые выборочные характеристик. На вход подаётся \pyf{n} - размер генерируемых данных; \pyf{dist} - распределение, по которому генерируются данные; \pyf{**param} - параметры распределения.\\\\
        \pyf{show_esstimates(dist,real_values,**params)} - функция - обёртка над\\
        \pyf{get_five_values}. Дополнительный параметр \pyf{real_values} - кортеж теоретических значений для распределения \pyf{dist}.
        
    \section{Результаты}
    Ниже представлены результаты подсчёта характеристик, с указанием погрешностей и теоретических значений.
    \begin{figure}[H]
    \centering
    \includegraphics[width=0.8\linewidth]{1.png}
    \caption{$N(0,1)$}
    \end{figure}\\

    \begin{figure}[H]
    \centering
    \includegraphics[width=0.8\linewidth]{2.png}
    \caption{$C(0,1)$}
    \end{figure}\\

    \begin{figure}[H]
    \centering
    \includegraphics[width=0.8\linewidth]{3.png}
    \caption{$L(0,\frac{1}{\sqrt{2}})$}
    \end{figure}\\

    \begin{figure}[H]
    \centering
    \includegraphics[width=0.8\linewidth]{4.png}
    \caption{$P(5)$}
    \end{figure}\\

    \begin{figure}[H]
    \centering
    \includegraphics[width=0.8\linewidth]{5.png}
    \caption{$U(-\sqrt{3},\sqrt{3})$}
    \end{figure}\\
    
    \section{Обсуждение}
    Из представленных выше таблиц видно, что если характеристика определена для распределения, то с увеличением размера выборки - её среднее стремится к теоретическому значению, а дисперсия уменьшается.\\\\
    Для распределения Коши наблюдается следующее - выборочное среднее для сгенерированных данных по нему данных отказывается сходиться при увеличении $n$. При этом медиана лишена этого недостатка. Это связано с тем, что последняя характеристика менее чувствительна к выбросам, чем среднее арифметическое.
    
    \section{Выводы}
    \begin{enumerate}
        \item С увеличением размера выборки, характеристики сходятся к теоретическим (если таковы определены).
        \item Среднее арифметическое не всегда хорошо описывает статистическую зависимость, так как оно подвержено выбросам в данных.
    \end{enumerate}
    \section{Список литературы}
    \begin{enumerate}
        \item Документация \texttt{scipy} -  https://docs.scipy.org/doc/scipy/
        \item Документация \texttt{seaborn} -  https://seaborn.pydata.org/
    \end{enumerate}
    \section{Приложение}
    https://github.com/haskell-md2/MatStat/blob/main/lab1-4/lab1-4.ipynb - Параграф "Лабораторная 2"
\end{document}
